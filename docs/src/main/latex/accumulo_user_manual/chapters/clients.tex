
% Licensed to the Apache Software Foundation (ASF) under one or more
% contributor license agreements. See the NOTICE file distributed with
% this work for additional information regarding copyright ownership.
% The ASF licenses this file to You under the Apache License, Version 2.0
% (the "License"); you may not use this file except in compliance with
% the License. You may obtain a copy of the License at
%
%     http://www.apache.org/licenses/LICENSE-2.0
%
% Unless required by applicable law or agreed to in writing, software
% distributed under the License is distributed on an "AS IS" BASIS,
% WITHOUT WARRANTIES OR CONDITIONS OF ANY KIND, either express or implied.
% See the License for the specific language governing permissions and
% limitations under the License.

\chapter{Writing Accumulo Clients}

\section{Running Client Code}

There are multiple ways to run Java code that uses Accumulo. Below is a list
of the different ways to execute client code.

\begin{itemize} 
  \item using java executable 
  \item using the accumulo script
  \item using the tool script 
\end{itemize}

In order to run client code written to run against Accumulo, you will need to
include the jars that Accumulo depends on in your classpath. Accumulo client
code depends on Hadoop and Zookeeper. For Hadoop add the hadoop core jar, all
of the jars in the Hadoop lib directory, and the conf directory to the
classpath. For Zookeeper 3.3 you only need to add the Zookeeper jar, and not
what is in the Zookeeper lib directory. You can run the following command on a
configured Accumulo system to see what its using for its classpath.

\small 
\begin{verbatim} 
$ACCUMULO_HOME/bin/accumulo classpath 
\end{verbatim}
\normalsize

Another option for running your code is to put a jar file in
\texttt{\$ACCUMULO\_HOME/lib/ext}. After doing this you can use the accumulo
script to execute your code. For example if you create a jar containing the
class com.foo.Client and placed that in lib/ext, then you could use the command
\texttt{\$ACCUMULO\_HOME/bin/accumulo com.foo.Client} to execute your code.

If you are writing map reduce job that access Accumulo, then you can use the
bin/tool.sh script to run those jobs. See the map reduce example.

\section{Connecting}

All clients must first identify the Accumulo instance to which they will be
communicating. Code to do this is as follows:

\small
\begin{verbatim}
String instanceName = "myinstance";
String zooServers = "zooserver-one,zooserver-two"
Instance inst = new ZooKeeperInstance(instanceName, zooServers);

Connector conn = inst.getConnector("user", new PasswordToken("passwd"));
\end{verbatim}
\normalsize

\section{Writing Data}

Data are written to Accumulo by creating Mutation objects that represent all the
changes to the columns of a single row. The changes are made atomically in the
TabletServer. Clients then add Mutations to a BatchWriter which submits them to
the appropriate TabletServers.

Mutations can be created thus:

\small
\begin{verbatim}
Text rowID = new Text("row1");
Text colFam = new Text("myColFam");
Text colQual = new Text("myColQual");
ColumnVisibility colVis = new ColumnVisibility("public");
long timestamp = System.currentTimeMillis();

Value value = new Value("myValue".getBytes());

Mutation mutation = new Mutation(rowID);
mutation.put(colFam, colQual, colVis, timestamp, value);
\end{verbatim}
\normalsize

\subsection{BatchWriter}
The BatchWriter is highly optimized to send Mutations to multiple TabletServers
and automatically batches Mutations destined for the same TabletServer to
amortize network overhead. Care must be taken to avoid changing the contents of
any Object passed to the BatchWriter since it keeps objects in memory while
batching.

Mutations are added to a BatchWriter thus:

\small
\begin{verbatim}
long memBuf = 1000000L; // bytes to store before sending a batch
long timeout = 1000L; // milliseconds to wait before sending
int numThreads = 10;

BatchWriter writer =
    conn.createBatchWriter("table", memBuf, timeout, numThreads)

writer.add(mutation);

writer.close();
\end{verbatim}
\normalsize

An example of using the batch writer can be found at\\
accumulo/docs/examples/README.batch

\section{Reading Data}

Accumulo is optimized to quickly retrieve the value associated with a given key, and
to efficiently return ranges of consecutive keys and their associated values.

\subsection{Scanner}

To retrieve data, Clients use a Scanner, which acts like an Iterator over
keys and values. Scanners can be configured to start and stop at particular keys, and
to return a subset of the columns available.

\small
\begin{verbatim}
// specify which visibilities we are allowed to see
Authorizations auths = new Authorizations("public");

Scanner scan =
    conn.createScanner("table", auths);

scan.setRange(new Range("harry","john"));
scan.fetchFamily("attributes");

for(Entry<Key,Value> entry : scan) {
    String row = entry.getKey().getRow();
    Value value = entry.getValue();
}
\end{verbatim}
\normalsize

\subsection{Isolated Scanner}

Accumulo supports the ability to present an isolated view of rows when
scanning. There are three possible ways that a row could change in Accumulo :

\begin{itemize}
 \item a mutation applied to a table
 \item iterators executed as part of a minor or major compaction
 \item bulk import of new files
\end{itemize}

Isolation guarantees that either all or none of the changes made by these
operations on a row are seen. Use the IsolatedScanner to obtain an isolated
view of an Accumulo table. When using the regular scanner it is possible to see
a non isolated view of a row. For example if a mutation modifies three
columns, it is possible that you will only see two of those modifications.
With the isolated scanner either all three of the changes are seen or none.

The IsolatedScanner buffers rows on the client side so a large row will not
crash a tablet server. By default rows are buffered in memory, but the user
can easily supply their own buffer if they wish to buffer to disk when rows are
large.

For an example, look at the following\\ 
\texttt{examples/simple/src/main/java/org/apache/accumulo/examples/simple/isolation/InterferenceTest.java}

\subsection{BatchScanner}

For some types of access, it is more efficient to retrieve several ranges
simultaneously. This arises when accessing a set of rows that are not consecutive
whose IDs have been retrieved from a secondary index, for example.

The BatchScanner is configured similarly to the Scanner; it can be configured to
retrieve a subset of the columns available, but rather than passing a single Range,
BatchScanners accept a set of Ranges. It is important to note that the keys returned
by a BatchScanner are not in sorted order since the keys streamed are from multiple
TabletServers in parallel.

\small
\begin{verbatim}
ArrayList<Range> ranges = new ArrayList<Range>();
// populate list of ranges ...

BatchScanner bscan =
    conn.createBatchScanner("table", auths, 10);

bscan.setRanges(ranges);
bscan.fetchFamily("attributes");

for(Entry<Key,Value> entry : scan)
    System.out.println(entry.getValue());
\end{verbatim}
\normalsize

An example of the BatchScanner can be found at\\
accumulo/docs/examples/README.batch

