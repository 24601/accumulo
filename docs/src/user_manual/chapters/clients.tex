\chapter{Writing Accumulo Clients}

All clients must first identify the Accumulo instance to which they will be
communicating. Code to do this is as follows:

\small
\begin{verbatim}
String instanceName = "myinstance";
String zooServers = "zooserver-one,zooserver-two"
Instance inst = new ZooKeeperInstance(instanceName, zooServers);

Connector conn = inst.getConnector("user", "passwd");
\end{verbatim}
\normalsize

\section{Writing Data}

Data are written to Accumulo by creating Mutation objects that represent all the
changes to the columns of a single row. The changes are made atomically in the
TabletServer. Clients then add Mutations to a BatchWriter which submits them to
the appropriate TabletServers.

Mutations can be created thus:

\small
\begin{verbatim}
Text rowID = new Text("row1");
Text colFam = new Text("myColFam");
Text colQual = new Text("myColQual");
ColumnVisibility colVis = new ColumnVisibility("public");
long timestamp = System.currentTimeMillis();

Value value = new Value("myValue".getBytes());

Mutation mutation = new Mutation(rowID);
mutation.put(colFam, colQual, colVis, timestamp, value);
\end{verbatim}
\normalsize

\subsection{BatchWriter}
The BatchWriter is highly optimized to send Mutations to multiple TabletServers
and automatically batches Mutations destined for the same TabletServer to
amortize network overhead. Care must be taken to avoid changing the contents of
any Object passed to the BatchWriter since it keeps objects in memory while
batching.

Mutations are added to a BatchWriter thus:

\small
\begin{verbatim}
long memBuf = 1000000L; // bytes to store before sending a batch
long timeout = 1000L; // milliseconds to wait before sending
int numThreads = 10;

BatchWriter writer =
    conn.createBatchWriter("table", memBuf, timeout, numThreads)

writer.add(mutation);

writer.close();
\end{verbatim}
\normalsize

An example of using the batch writer can be found at\\
accumulo/docs/examples/README.batch

\section{Reading Data}

Accumulo is optimized to quickly retrieve the value associated with a given key, and
to efficiently return ranges of consecutive keys and their associated values.

\subsection{Scanner}

To retrieve data, Clients use a Scanner, which provides acts like an Iterator over
keys and values. Scanners can be configured to start and stop at particular keys, and
to return a subset of the columns available.

\small
\begin{verbatim}
// specify which visibilities we are allowed to see
Authorizations auths = new Authorizations("public");

Scanner scan =
    conn.createScanner("table", auths);

scan.setRange(new Range("harry","john"));
scan.fetchFamily("attributes");

for(Entry<Key,Value> entry : scan) {
    String row = e.getKey().getRow();
    Value value = e.getValue();
}
\end{verbatim}
\normalsize

\subsection{BatchScanner}

For some types of access, it is more efficient to retrieve several ranges
simultaneously. This arises when accessing a set of rows that are not consecutive
whose IDs have been retrieved from a secondary index, for example.

The BatchScanner is configured similarly to the Scanner; it can be configured to
retrieve a subset of the columns available, but rather than passing a single Range,
BatchScanners accept a set of Ranges. It is important to note that the keys returned
by a BatchScanner are not in sorted order since the keys streamed are from multiple
TabletServers in parallel.

\small
\begin{verbatim}
ArrayList<Range> ranges = new ArrayList<Range>();
// populate list of ranges ...

BatchScanner bscan =
    conn.createBatchScanner("table", auths, 10);

bscan.setRanges(ranges);
bscan.fetchFamily("attributes");

for(Entry<Key,Value> entry : scan)
    System.out.println(e.getValue());
\end{verbatim}
\normalsize

An example of the BatchScanner can be found at\\
accumulo/docs/examples/README.batch

