
% Licensed to the Apache Software Foundation (ASF) under one or more
% contributor license agreements.  See the NOTICE file distributed with
% this work for additional information regarding copyright ownership.
% The ASF licenses this file to You under the Apache License, Version 2.0
% (the "License"); you may not use this file except in compliance with
% the License.  You may obtain a copy of the License at
%
%     http://www.apache.org/licenses/LICENSE-2.0
%
% Unless required by applicable law or agreed to in writing, software
% distributed under the License is distributed on an "AS IS" BASIS,
% WITHOUT WARRANTIES OR CONDITIONS OF ANY KIND, either express or implied.
% See the License for the specific language governing permissions and
% limitations under the License.

\chapter{Table Configuration}

Accumulo tables have a few options that can be configured to alter the default
behavior of Accumulo as well as improve performance based on the data stored.
These include locality groups, constraints, and iterators.

\section{Locality Groups}
Accumulo supports storing of sets of column families separately on disk to allow
clients to scan over columns that are frequently used together efficient and to avoid
scanning over column families that are not requested. After a locality group is set
Scanner and BatchScanner operations will automatically take advantage of them
whenever the fetchColumnFamilies() method is used.

By default tables place all column families into the same ``default" locality group.
Additional locality groups can be configured anytime via the shell or
programmatically as follows:

\subsection{Managing Locality Groups via the Shell}

\small
\begin{verbatim}
usage: setgroups <group>=<col fam>{,<col fam>}{ <group>=<col fam>{,<col
fam>}} [-?] -t <table>

user@myinstance mytable> setgroups -t mytable group_one=colf1,colf2

user@myinstance mytable> getgroups -t mytable
group_one=colf1,colf2
\end{verbatim}
\normalsize

\subsection{Managing Locality Groups via the Client API}

\small
\begin{verbatim}
Connector conn;

HashMap<String,Set<Text>> localityGroups =
    new HashMap<String, Set<Text>>();

HashSet<Text> metadataColumns = new HashSet<Text>();
metadataColumns.add(new Text("domain"));
metadataColumns.add(new Text("link"));

HashSet<Text> contentColumns = new HashSet<Text>();
contentColumns.add(new Text("body"));
contentColumns.add(new Text("images"));

localityGroups.put("metadata", metadataColumns);
localityGroups.put("content", contentColumns);

conn.tableOperations().setLocalityGroups("mytable", localityGroups);

// existing locality groups can be obtained as follows
Map<String, Set<Text>> groups =
    conn.tableOperations().getLocalityGroups("mytable");
\end{verbatim}
\normalsize

The assignment of Column Families to Locality Groups can be changed anytime. The
physical movement of column families into their new locality groups takes place via
the periodic Major Compaction process that takes place continuously in the
background. Major Compaction can also be scheduled to take place immediately
through the shell:

\small
\begin{verbatim}
user@myinstance mytable> compact -t mytable
\end{verbatim}
\normalsize

\section{Constraints}

Accumulo supports constraints applied on mutations at insert time. This can be
used to disallow certain inserts according to a user defined policy. Any mutation
that fails to meet the requirements of the constraint is rejected and sent back to the
client.

Constraints can be enabled by setting a table property as follows:

\small
\begin{verbatim}
user@myinstance mytable> config -t mytable -s table.constraint.1=com.test.ExampleConstraint
user@myinstance mytable> config -t mytable -s table.constraint.2=com.test.AnotherConstraint
user@myinstance mytable> config -t mytable -f constraint
---------+--------------------------------+----------------------------
SCOPE    | NAME                           | VALUE
---------+--------------------------------+----------------------------
table    | table.constraint.1............ | com.test.ExampleConstraint
table    | table.constraint.2............ | com.test.AnotherConstraint
---------+--------------------------------+----------------------------
\end{verbatim}
\normalsize

Currently there are no general-purpose constraints provided with the Accumulo
distribution. New constraints can be created by writing a Java class that implements
the org.apache.accumulo.core.constraints.Constraint interface.

To deploy a new constraint, create a jar file containing the class implementing the
new constraint and place it in the lib directory of the Accumulo installation. New
constraint jars can be added to Accumulo and enabled without restarting but any
change to an existing constraint class requires Accumulo to be restarted.

An example of constraints can be found in\\
\texttt{accumulo/docs/examples/README.constraints} with corresponding code under\\
\texttt{accumulo/src/examples/main/java/accumulo/examples/constraints} .

\section{Bloom Filters}
As mutations are applied to a Accumulo table, several files are created per tablet. If
bloom filters are enabled, Accumulo will create and load a small data structure into
memory to determine whether a file contains a given key before opening the file.
This can speed up lookups considerably.

To enable bloom filters, enter the following command in the Shell:

\small
\begin{verbatim}
user@myinstance> config -t mytable -s table.bloom.enabled=true
\end{verbatim}
\normalsize

An extensive example of using Bloom Filters can be found at\\
\texttt{accumulo/docs/examples/README.bloom} .

\section{Iterators}
Iterators provide a modular mechanism for adding functionality to be executed by
TabletServers when scanning or compacting data. This allows users to efficiently
summarize, filter, and aggregate data. In fact, the built-in features of cell-level
security and age-off are implemented using Iterators.

\subsection{Setting Iterators via the Shell}

\small
\begin{verbatim}
usage: setiter [-?] -agg | -class <name> | -filter | -nolabel | 
-regex | -vers [-majc] [-minc] [-n <itername>] -p <pri> [-scan] 
[-t <table>]

user@myinstance mytable> setiter -t mytable -scan -p 10 -n myiter
\end{verbatim}
\normalsize

\subsection{Setting Iterators Programmatically}

\small
\begin{verbatim}
scanner.setScanIterators(
    15, // priority
    "com.company.MyIterator", // class name
    "myiter"); // name this iterator
\end{verbatim}
\normalsize

Some iterators take additional parameters from client code, as in the following
example:

\small
\begin{verbatim}
bscan.setIteratorOption(
    "myiter", // iterator reference
    "myoptionname",
    "myoptionvalue");
\end{verbatim}
\normalsize

Tables support separate Iterator settings to be applied at scan time, upon minor
compaction and upon major compaction. For most uses, tables will have identical
iterator settings for all three to avoid inconsistent results.

\section{Versioning Iterators and Timestamps}

Accumulo provides the capability to manage versioned data through the use of
timestamps within the Key. If a timestamp is not specified in the key created by the
client then the system will set the timestamp to the current time. Two keys with
identical rowIDs and columns but different timestamps are considered two versions
of the same key. If two inserts are made into accumulo with the same rowID,
column, and timestamp, then the behavior is non-deterministic.

Timestamps are sorted in descending order, so the most recent data comes first.
Accumulo can be configured to return the top k versions, or versions later than a
given date. The default is to return the one most recent version.

The version policy can be changed by changing the VersioningIterator options for a
table as follows:

\small
\begin{verbatim}
user@myinstance mytable> config -t mytable -s
table.iterator.scan.vers.opt.maxVersions=3

user@myinstance mytable> config -t mytable -s
table.iterator.minc.vers.opt.maxVersions=3

user@myinstance mytable> config -t mytable -s
table.iterator.majc.vers.opt.maxVersions=3
\end{verbatim}
\normalsize

\subsection{Logical Time}

Accumulo 1.2 introduces the concept of logical time. This ensures that timestamps
set by accumulo always move forward. This helps avoid problems caused by
TabletServers that have different time settings. The per tablet counter gives unique
one up time stamps on a per mutation basis. When using time in milliseconds, if two
things arrive within the same millisecond then both receive the same timestamp.

A table can be configured to use logical timestamps at creation time as follows:

\small
\begin{verbatim}
user@myinstance> createtable -tl logical
\end{verbatim}
\normalsize

\subsection{Deletes}
Deletes are special keys in accumulo that get sorted along will all the other data.
When a delete key is inserted, accumulo will not show anything that has a
timestamp less than or equal to the delete key. During major compaction, any keys
older than a delete key are omitted from the new file created, and the omitted keys
are removed from disk as part of the regular garbage collection process.

\section{Filtering Iterators}
When scanning over a set of key-value pairs it is possible to apply an arbitrary
filtering policy through the use of a FilteringIterator. These types of iterators return
only key-value pairs that satisfy the filter logic. Accumulo has two built-in filtering
iterators that can be configured on any table: AgeOff and RegEx. More can be added
by writing a Java class that implements the\\
org.apache.accumulo.core.iterators.filter.Filter interface.

To configure the AgeOff filter to remove data older than a certain date or a fixed
amount of time from the present. The following example sets a table to delete
everything inserted over 30 seconds ago:

\small
\begin{verbatim}
user@myinstance> createtable filtertest
user@myinstance filtertest> setiter -t filtertest -scan -minc -majc -p
10 -n myfilter -filter

FilteringIterator uses Filters to accept or reject key/value pairs
----------> entering options: <filterPriorityNumber>
<ageoff|regex|filterClass>

----------> set org.apache.accumulo.core.iterators.FilteringIterator option
(<name> <value>, hit enter to skip): 0 ageoff

----------> set org.apache.accumulo.core.iterators.FilteringIterator option
(<name> <value>, hit enter to skip):
AgeOffFilter removes entries with timestamps more than <ttl>
milliseconds old

----------> set org.apache.accumulo.core.iterators.filter.AgeOffFilter parameter
currentTime, if set, use the given value as the absolute time in
milliseconds as the current time of day:

----------> set org.apache.accumulo.core.iterators.filter.AgeOffFilter parameter
ttl, time to live (milliseconds): 30000

user@myinstance filtertest>
user@myinstance filtertest> scan
user@myinstance filtertest> insert foo a b c
insert successful
user@myinstance filtertest> scan
foo a:b [] c

... wait 30 seconds ...

user@myinstance filtertest> scan
user@myinstance filtertest>
\end{verbatim}
\normalsize

To see the iterator settings for a table, use:

\small
\begin{verbatim}
user@example filtertest> config -t filtertest -f iterator
---------+------------------------------------------+------------------
SCOPE    | NAME                                     | VALUE
---------+------------------------------------------+------------------
table    | table.iterator.majc.myfilter ........... |
10,org.apache.accumulo.core.iterators.FilteringIterator
table    | table.iterator.majc.myfilter.opt.0 ..... |
org.apache.accumulo.core.iterators.filter.AgeOffFilter
table    | table.iterator.majc.myfilter.opt.0.ttl . | 30000
table    | table.iterator.minc.myfilter ........... |
10,org.apache.accumulo.core.iterators.FilteringIterator
table    | table.iterator.minc.myfilter.opt.0 ..... |
org.apache.accumulo.core.iterators.filter.AgeOffFilter
table    | table.iterator.minc.myfilter.opt.0.ttl . | 30000
table    | table.iterator.scan.myfilter ........... |
10,org.apache.accumulo.core.iterators.FilteringIterator
table    | table.iterator.scan.myfilter.opt.0 ..... |
org.apache.accumulo.core.iterators.filter.AgeOffFilter
table    | table.iterator.scan.myfilter.opt.0.ttl . | 30000
---------+------------------------------------------+------------------
\end{verbatim}
\normalsize

\section{Aggregating Iterators}

Accumulo allows aggregating iterators to be configured on tables and column
families. When an aggregating iterator is set, the iterator is applied across the values
associated with any keys that share rowID, column family, and column qualifier.
This is similar to the reduce step in MapReduce, which applied some function to all
the values associated with a particular key.

For example, if an aggregating iterator were configured on a table and the following
mutations were inserted:

\small
\begin{verbatim}
Row     Family Qualifier Timestamp  Value
rowID1  colfA  colqA     20100101   1
rowID1  colfA  colqA     20100102   1
\end{verbatim}
\normalsize

The table would reflect only one aggregate value:

\small
\begin{verbatim}
rowID1  colfA  colqA     -          2
\end{verbatim}
\normalsize

Aggregating iterators can be enabled for a table as follows:

\small
\begin{verbatim}
user@myinstance> createtable perDayCounts -a
day=org.apache.accumulo.core.iterators.aggregation.StringSummation

user@myinstance perDayCounts> insert row1 day 20080101 1
user@myinstance perDayCounts> insert row1 day 20080101 1
user@myinstance perDayCounts> insert row1 day 20080103 1
user@myinstance perDayCounts> insert row2 day 20080101 1
user@myinstance perDayCounts> insert row3 day 20080101 1

user@myinstance perDayCounts> scan
row1 day:20080101 [] 2
row1 day:20080103 [] 1
row2 day:20080101 [] 2
\end{verbatim}
\normalsize

Accumulo includes the following aggregators:
\begin{itemize}
\item{\textbf{LongSummation}: expects values of type long and adds them.}
\item{\textbf{StringSummation}: expects numbers represented as strings and adds them.}
\item{\textbf{StringMax}: expects numbers as strings and retains the maximum number inserted.}
\item{\textbf{StringMin}: expects numbers as strings and retains the minimum number inserted.}
\end{itemize}

Additional Aggregators can be added by creating a Java class that implements\\
\textbf{org.apache.accumulo.core.iterators.aggregation.Aggregator} and adding a jar containing that
class to Accumulo's lib directory.

An example of an aggregator can be found under\\
accumulo/src/examples/main/java/accumulo/examples/aggregation/SortedSetAggregator.java

