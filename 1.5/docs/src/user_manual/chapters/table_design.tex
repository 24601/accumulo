
% Licensed to the Apache Software Foundation (ASF) under one or more
% contributor license agreements.  See the NOTICE file distributed with
% this work for additional information regarding copyright ownership.
% The ASF licenses this file to You under the Apache License, Version 2.0
% (the "License"); you may not use this file except in compliance with
% the License.  You may obtain a copy of the License at
%
%     http://www.apache.org/licenses/LICENSE-2.0
%
% Unless required by applicable law or agreed to in writing, software
% distributed under the License is distributed on an "AS IS" BASIS,
% WITHOUT WARRANTIES OR CONDITIONS OF ANY KIND, either express or implied.
% See the License for the specific language governing permissions and
% limitations under the License.

\chapter{Table Design}

\section{Basic Table}

Since Accumulo tables are sorted by row ID, each table can be thought of as being
indexed by the row ID. Lookups performed row ID can be executed quickly, by doing
a binary search, first across the tablets, and then within a tablet. Clients should
choose a row ID carefully in order to support their desired application. A simple rule
is to select a unique identifier as the row ID for each entity to be stored and assign
all the other attributes to be tracked to be columns under this row ID. For example,
if we have the following data in a comma-separated file:

\small
\begin{verbatim}
    userid,age,address,account-balance
\end{verbatim}
\normalsize

We might choose to store this data using the userid as the rowID and the rest of the
data in column families:

\small
\begin{verbatim}
Mutation m = new Mutation(new Text(userid));
m.put(new Text("age"), age);
m.put(new Text("address"), address);
m.put(new Text("balance"), account_balance);

writer.add(m);
\end{verbatim}
\normalsize

We could then retrieve any of the columns for a specific userid by specifying the
userid as the range of a scanner and fetching specific columns:

\small
\begin{verbatim}
Range r = new Range(userid, userid); // single row
Scanner s = conn.createScanner("userdata", auths);
s.setRange(r);
s.fetchColumnFamily(new Text("age"));

for(Entry<Key,Value> entry : s)
    System.out.println(entry.getValue().toString());
\end{verbatim}
\normalsize

\section{RowID Design}

Often it is necessary to transform the rowID in order to have rows ordered in a way
that is optimal for anticipated access patterns. A good example of this is reversing
the order of components of internet domain names in order to group rows of the
same parent domain together:

\small
\begin{verbatim}
com.google.code
com.google.labs
com.google.mail
com.yahoo.mail
com.yahoo.research
\end{verbatim}
\normalsize

Some data may result in the creation of very large rows - rows with many columns.
In this case the table designer may wish to split up these rows for better load
balancing while keeping them sorted together for scanning purposes. This can be
done by appending a random substring at the end of the row:

\small
\begin{verbatim}
com.google.code_00
com.google.code_01
com.google.code_02
com.google.labs_00
com.google.mail_00
com.google.mail_01
\end{verbatim}
\normalsize

It could also be done by adding a string representation of some period of time such as date to the week
or month:

\small
\begin{verbatim}
com.google.code_201003
com.google.code_201004
com.google.code_201005
com.google.labs_201003
com.google.mail_201003
com.google.mail_201004
\end{verbatim}
\normalsize

Appending dates provides the additional capability of restricting a scan to a given
date range.

\section{Indexing}
In order to support lookups via more than one attribute of an entity, additional
indexes can be built. However, because Accumulo tables can support any number of
columns without specifying them beforehand, a single additional index will often
suffice for supporting lookups of records in the main table. Here, the index has, as
the rowID, the Value or Term from the main table, the column families are the same,
and the column qualifier of the index table contains the rowID from the main table.

\begin{center}
$\begin{array}{|c|c|c|c|c|c|} \hline
\multicolumn{5}{|c|}{\mbox{Key}} & \multirow{3}{*}{\mbox{Value}}\\ \cline{1-5}
\multirow{2}{*}{\mbox{Row ID}}& \multicolumn{3}{|c|}{\mbox{Column}} & \multirow{2}{*}{\mbox{Timestamp}} & \\ \cline{2-4}
& \mbox{Family} & \mbox{Qualifier} & \mbox{Visibility} & & \\ \hline \hline
\mbox{Term} & \mbox{Field Name} & \mbox{MainRowID} & & &\\ \hline
\end{array}$
\end{center}

Note: We store rowIDs in the column qualifier rather than the Value so that we can
have more than one rowID associated with a particular term within the index. If we
stored this in the Value we would only see one of the rows in which the value
appears since Accumulo is configured by default to return the one most recent
value associated with a key.

Lookups can then be done by scanning the Index Table first for occurrences of the
desired values in the columns specified, which returns a list of row ID from the main
table. These can then be used to retrieve each matching record, in their entirety, or a
subset of their columns, from the Main Table.

To support efficient lookups of multiple rowIDs from the same table, the Accumulo
client library provides a BatchScanner. Users specify a set of Ranges to the
BatchScanner, which performs the lookups in multiple threads to multiple servers
and returns an Iterator over all the rows retrieved. The rows returned are NOT in
sorted order, as is the case with the basic Scanner interface.

\small
\begin{verbatim}
// first we scan the index for IDs of rows matching our query

Text term = new Text("mySearchTerm");

HashSet<Text> matchingRows = new HashSet<Text>();

Scanner indexScanner = createScanner("index", auths);
indexScanner.setRange(new Range(term, term));

// we retrieve the matching rowIDs and create a set of ranges
for(Entry<Key,Value> entry : indexScanner)
    matchingRows.add(new Text(entry.getKey().getColumnQualifier()));

// now we pass the set of rowIDs to the batch scanner to retrieve them
BatchScanner bscan = conn.createBatchScanner("table", auths, 10);

bscan.setRanges(matchingRows);
bscan.fetchFamily("attributes");

for(Entry<Key,Value> entry : scan)
    System.out.println(entry.getValue());
\end{verbatim}
\normalsize

One advantage of the dynamic schema capabilities of Accumulo is that different
fields may be indexed into the same physical table. However, it may be necessary to
create different index tables if the terms must be formatted differently in order to
maintain proper sort order. For example, real numbers must be formatted
differently than their usual notation in order to be sorted correctly. In these cases,
usually one index per unique data type will suffice.

\section{Entity-Attribute and Graph Tables}

Accumulo is ideal for storing entities and their attributes, especially of the
attributes are sparse. It is often useful to join several datasets together on common
entities within the same table. This can allow for the representation of graphs,
including nodes, their attributes, and connections to other nodes.

Rather than storing individual events, Entity-Attribute or Graph tables store
aggregate information about the entities involved in the events and the
relationships between entities. This is often preferrable when single events aren't
very useful and when a continuously updated summarization is desired.

The physical schema for an entity-attribute or graph table is as follows:

\begin{center}
$\begin{array}{|c|c|c|c|c|c|} \hline
\multicolumn{5}{|c|}{\mbox{Key}} & \multirow{3}{*}{\mbox{Value}}\\ \cline{1-5}
\multirow{2}{*}{\mbox{Row ID}}& \multicolumn{3}{|c|}{\mbox{Column}} & \multirow{2}{*}{\mbox{Timestamp}} & \\ \cline{2-4}
& \mbox{Family} & \mbox{Qualifier} & \mbox{Visibility} & & \\ \hline \hline
\mbox{EntityID} & \mbox{Attribute Name} & \mbox{Attribute Value} & & & \mbox{Weight} \\ \hline
\mbox{EntityID} & \mbox{Edge Type} & \mbox{Related EntityID} & & & \mbox{Weight} \\ \hline
\end{array}$
\end{center}

For example, to keep track of employees, managers and products the following
entity-attribute table could be used. Note that the weights are not always necessary
and are set to 0 when not used.

$\begin{array}{llll}
\bf{RowID} & \bf{Column Family} & \bf{Column Qualifier} & \bf{Value} \\
\\
E001 & name & bob & 0 \\
E001 & department & sales & 0 \\
E001 & hire\_date & 20030102 & 0 \\
E001 & units\_sold & P001 & 780 \\
\\
E002 & name & george & 0 \\
E002 & department & sales & 0 \\
E002 & manager\_of & E001 & 0 \\
E002 & manager\_of & E003 & 0 \\
\\
E003 & name & harry & 0 \\
E003 & department & accounts\_recv & 0 \\
E003 & hire\_date & 20000405 & 0 \\
E003 & units\_sold & P002 & 566 \\
E003 & units\_sold & P001 & 232 \\
\\
P001 & product\_name & nike\_airs & 0 \\
P001 & product\_type & shoe & 0 \\
P001 & in\_stock & germany & 900 \\
P001 & in\_stock & brazil & 200 \\
\\
P002 & product\_name & basic\_jacket & 0 \\
P002 & product\_type & clothing & 0 \\
P002 & in\_stock & usa & 3454 \\
P002 & in\_stock & germany & 700 \\
\end{array}$
\vspace{5mm}

To allow efficient updating of edge weights, an aggregating iterator can be
configured to add the value of all mutations applied with the same key. These types
of tables can easily be created from raw events by simply extracting the entities,
attributes, and relationships from individual events and inserting the keys into
Accumulo each with a count of 1. The aggregating iterator will take care of
maintaining the edge weights.

\section{Document-Partitioned Indexing}

Using a simple index as described above works well when looking for records that
match one of a set of given criteria. When looking for records that match more than
one criterion simultaneously, such as when looking for documents that contain all of
the words `the' and `white' and `house', there are several issues.

First is that the set of all records matching any one of the search terms must be sent
to the client, which incurs a lot of network traffic. The second problem is that the
client is responsible for performing set intersection on the sets of records returned
to eliminate all but the records matching all search terms. The memory of the client
may easily be overwhelmed during this operation.

For these reasons Accumulo includes support for a scheme known as sharded
indexing, in which these set operations can be performed at the TabletServers and
decisions about which records to include in the result set can be made without
incurring network traffic.

This is accomplished via partitioning records into bins that each reside on at most
one TabletServer, and then creating an index of terms per record within each bin as
follows:

\begin{center}
$\begin{array}{|c|c|c|c|c|c|} \hline
\multicolumn{5}{|c|}{\mbox{Key}} & \multirow{3}{*}{\mbox{Value}}\\ \cline{1-5}
\multirow{2}{*}{\mbox{Row ID}}& \multicolumn{3}{|c|}{\mbox{Column}} & \multirow{2}{*}{\mbox{Timestamp}} & \\ \cline{2-4}
& \mbox{Family} & \mbox{Qualifier} & \mbox{Visibility} & & \\ \hline \hline
\mbox{BinID} & \mbox{Term} & \mbox{DocID} & & & \mbox{Weight} \\ \hline
\end{array}$
\end{center}

Documents or records are mapped into bins by a user-defined ingest application. By
storing the BinID as the RowID we ensure that all the information for a particular
bin is contained in a single tablet and hosted on a single TabletServer since
Accumulo never splits rows across tablets. Storing the Terms as column families
serves to enable fast lookups of all the documents within this bin that contain the
given term.

Finally, we perform set intersection operations on the TabletServer via a special
iterator called the Intersecting Iterator. Since documents are partitioned into many
bins, a search of all documents must search every bin. We can use the BatchScanner
to scan all bins in parallel. The Intersecting Iterator should be enabled on a
BatchScanner within user query code as follows:

\small
\begin{verbatim}
Text[] terms = {new Text("the"), new Text("white"), new Text("house")};

BatchScanner bs = conn.createBatchScanner(table, auths, 20);
IteratorSetting iter = new IteratorSetting(20, "ii", IntersectingIterator.class);
IntersectingIterator.setColumnFamilies(iter, terms);
bs.addScanIterator(iter);
bs.setRanges(Collections.singleton(new Range()));

for(Entry<Key,Value> entry : bs) {
    System.out.println(" " + entry.getKey().getColumnQualifier());
}
\end{verbatim}
\normalsize

This code effectively has the BatchScanner scan all tablets of a table, looking for
documents that match all the given terms. Because all tablets are being scanned for
every query, each query is more expensive than other Accumulo scans, which
typically involve a small number of TabletServers. This reduces the number of
concurrent queries supported and is subject to what is known as the `straggler'
problem in which every query runs as slow as the slowest server participating.

Of course, fast servers will return their results to the client which can display them
to the user immediately while they wait for the rest of the results to arrive. If the
results are unordered this is quite effective as the first results to arrive are as good
as any others to the user.

